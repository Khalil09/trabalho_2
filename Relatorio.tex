\documentclass[a4paper, 12pt]{article}
\usepackage[portuguese]{babel}
\usepackage[utf8]{inputenc}
\usepackage[T1]{fontenc}
\usepackage{amssymb}
\usepackage{indentfirst}
\usepackage{graphicx}
\usepackage{fancyhdr}
\usepackage{circuitikz}
\usepackage{tikz}
\usepackage{authblk}
\usepackage{geometry}
\usepackage[colorinlistoftodos]{todonotes}
\usepackage{caption}
\usepackage{amsmath}
\usepackage{algorithm}
\usepackage[noend]{algpseudocode}
\geometry{a4paper, total={170mm,257mm}, left=25mm, top=25mm, right=25mm, bottom=25mm}
\usetikzlibrary{positioning}
\DeclareGraphicsExtensions{.pdf,.png,.jpg}
\makeindex
\author[]{Renato Nobre - 15/0146698\\Khalil Carsten - 15/0134495}
\affil[]{Estrutura de Dados\\Universidade de Brasília}
\title{Relatório\\Trabalho 2}
\date{21 de Junho de 2016}
\pagestyle{fancy}
\fancyhf{}
\lhead{Trabalho 2}
\rhead{Universidade de Brasília\\IE - Departamento de Ciência da Computação}
\headsep = 40pt
\captionsetup{labelformat=empty}

\makeatletter
\def\BState{\State\hskip-\ALG@thistlm}
\makeatother

\begin{document}
	\begin{titlepage}

		\newcommand{\HRule}{\rule{\linewidth}{0.5mm}}
		\centering
        %\maketitle
	    \textsc{\LARGE Universidade de Brasília}\\[0.5cm]
	    \includegraphics{logo.jpg}\\[0.5cm]
		\textsc{\Large Instituto de Ciências Exatas}\\[0.5cm]
	    \textsc{\Large Departamento de Ciência da Computação}\\[0.5cm]
		\textsc{\Large Estrutura de Dados - Turma "A"}\\[0.5cm]
	    \HRule \\[0.4cm]
	    { \huge \bfseries Relatório - Trabalho 2}\\
	      \HRule \\[1.5cm]
	      	\begin{minipage}{0.4\textwidth}
	      		\begin{flushleft} \large
	      			\emph{Nome:}\\
	     			\emph{Renato Nobre}\\
	      			\emph{Khalil Carsten}\\
	      		\end{flushleft}
	      	\end{minipage}
	      	~
	      	\begin{minipage}{0.4\textwidth}
	      		\begin{flushright} \large
	      			\emph{Matrícula:}\\
	     			\textsc{15/0146698}\\
	        		\textsc{15/0132662}\\
	      		\end{flushright}
	      	\end{minipage}\\[2cm]
		\textsc{\large \centering 03 de Março de 2016}\\
	\end{titlepage}
	
	
	\section{Introdução}
		\textrm{Mancala é denominada à uma categoria de jogos com sua evidencia de criação no século 6 e 7 \emph{Anno Domini}, havendo evidencias de sua criação em Eritreia e Etiopia. O Mancala apresenta claramente, similaridades com a agricultura, e a ausencia de equipamentos que podem ser evidencia do inicio da civilização em si [4]. Uma versão moderna do jogo é o Kalah, usualmente jogado nos Estados Unidos e na Europa, onde há a confusão entre as nomeclaturas do jogo. Como outros jogos de tabuleiro, o mancala já serviu de diversos estudos, tanto psicológicos, como na ciência da computação[5].}
	
		\textrm{Em estrutura de dados, uma árvore é uma forma de organização hierárquica. Contendo, nós, raiz, ramos, folhas, e mais diversas terminologias para sua classificação. Tal estrutura é amplamente utilizada em classificações e tomada de decisões, como por exemplo no desenvolvimento de uma inteligência artificial básica para um jogo simples de tabuleiro.}
		
	\section{Implementação}
		\textrm{}
		

	\section{Conclusão}
		\textrm{}
		
	\section{Bibliografia}
	    [1] Árvores de Jogos, https://en.wikipedia.org/wiki/Game_tree
	
		[2] Prof. Eduardo Alchieri, Estrutura de Dados, Slides, Árvores, http://cic.unb.br/~alchieri/disciplinas/graduacao/ed/arvores.pdf 
		
		[3] Jogo Mancala Online, http://play-mancala.com
		
		[4] Mancala, https://en.wikipedia.org/wiki/Mancala
	
		[5] Gobet, F. (2009). '"Using a cognitive architecture for addressing the question of cognitive universals in cross-cultural psychology: The example of awalé". Journal of Cross-Cultural Psychology 40 (4): 627–648. doi:10.1177/0022022109335186
		
		
		
\end{document}