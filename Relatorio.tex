\documentclass[a4paper, 12pt]{article}
\usepackage[portuguese]{babel}
\usepackage[utf8]{inputenc}
\usepackage[T1]{fontenc}
\usepackage{indentfirst}
\usepackage{graphicx}
\usepackage{fancyhdr}
\usepackage{float}
\usepackage{authblk}
\usepackage{geometry}
\usepackage{caption}


\geometry{left=25mm, top=25mm, right=25mm, bottom=25mm}
\DeclareGraphicsExtensions{.pdf,.png,.jpg}
\pagestyle{fancy}
\fancyhf{}
\lhead{Trabalho 2}
\rhead{Universidade de Brasília\\IE - Departamento de Ciência da Computação}
\headsep = 40pt
\captionsetup{labelformat=empty}

\begin{document}
	\begin{titlepage}

		\newcommand{\HRule}{\rule{\linewidth}{0.5mm}}
		\centering
	    \textsc{\LARGE Universidade de Brasília}\\[0.5cm]
	    \includegraphics{logo.jpg}\\[0.5cm]
		\textsc{\Large Instituto de Ciências Exatas}\\[0.5cm]
	    \textsc{\Large Departamento de Ciência da Computação}\\[0.5cm]
		\textsc{\Large Estrutura de Dados - Turma ``A''}\\[0.5cm]
	    \HRule \\[0.4cm]
	    { \huge \bfseries Relatório - Trabalho 2}\\
	      \HRule \\[1.5cm]
	      	\begin{minipage}{0.4\textwidth}
	      		\begin{flushleft} \large
	      			\emph{Nome:}\\
	     			\emph{Renato Nobre}\\
	      			\emph{Khalil Carsten}\\
	      		\end{flushleft}
	      	\end{minipage}
	      	~
	      	\begin{minipage}{0.4\textwidth}
	      		\begin{flushright} \large
	      			\emph{Matrícula:}\\
	     			\textsc{15/0146698}\\
	        		\textsc{15/0132662}\\
	      		\end{flushright}
	      	\end{minipage}\\[2cm]
		\textsc{\large \centering 03 de Março de 2016}\\
	\end{titlepage}


	\section{Introdução}
		Mancala é denominada à uma categoria de jogos com sua evidencia de criação no século 6 e 7 \emph{Anno Domini}, havendo evidencias de sua criação em Eritreia e Etiópia. O Mancala apresenta claramente, similaridades com a agricultura, e a ausência de equipamentos que podem ser evidencia do inicio da civilização em si \cite{mancala}. Uma versão moderna do jogo é o Kalah, usualmente jogado nos Estados Unidos e na Europa, onde há a confusão entre as nomenclaturas do jogo. Como outros jogos de tabuleiro, o mancala já serviu de diversos estudos, tanto psicológicos, como na ciência da computação \cite{gobet}.

		O Kalah consiste de um tabuleiro de 14 cavidades, sendo 2 maiores denominadas Kahalas, e outras 12 cavidades menores, sendo 6 para cada jogador. A Kahala de cada jogador fica ao lado direito das cavidades menores, e o número de sementes define a pontuação do jogador.

		O estado inicial do jogo consiste em 4 sementes em cada cavidade dos jogadores, uma jogada é realizada quando o jogador escolhe uma das seis cavidades, retira todas as suas sementes e as distribui uma para cada cavidade no sentido anti-horário. A Kahala do jogador deve ser semeada também, entanto a Kahala do adversário deve ser ignorada. O jogo termina quando não há mais sementes em algum dos lados do tabuleiro. O vencedor é aquele que no final do jogo tiver mais sementes em sua Kahala. Regras adicionais são descritas detalhadamente dentro do programa.

		Em estrutura de dados, uma árvore é uma forma de organização hierárquica. Contendo, nós, raiz, ramos, folhas, e mais diversas terminologias para sua classificação. Tal estrutura é amplamente utilizada em classificações e tomada de decisões, como por exemplo no desenvolvimento de uma inteligência artificial básica para um jogo simples de tabuleiro.

		Uma árvore usualmente utilizada para esse propósito é a árvore genética, \emph{Game Tree}. Tal estrutura de árvore é montada de maneira a representar as possibilidades de jogadas de um jogador, a partir do estado do jogo. Os filhos de um nó representam todos os estados de jogada a partir da situação atual do jogo. Tal árvore pode ser criada com uma heurística para cada nó, e com uma função de avaliação \emph{Minimax}, retornar a melhor jogada.


	\section{Implementação}
		Com o propósito de entender por total o funcionamento e regras do jogo, um tempo foi gasto realizando seu estudo. Para isso foi utilizado uma versão online do jogo, a mesma versão nos forneceu as regras, a quais foram utilizadas na seção \emph{Regras}, do programa \cite{mancalaOnline}.

		O código é feito de maneira a tentar maximizar a separação da lógica da interface, contendo diversas funções para mostrar as mensagens na tela, disponibilizadas no começo do código, \emph{tabuleiro}, \emph{menu}, \emph{menu\_dific}. Tais funções servem respectivamente para mostrar o tabuleiro na tela, mostrar o menu principal e o menu de regras, e mostrar o menu após o modo \emph{Player vs IA} ser escolhido. Há também uma função \emph{popular}, que preenche o tabuleiro com os valores iniciais do jogo.

		\begin{figure}[H]
                \centering
                \includegraphics[scale=0.5]{Figura1.png}
                \caption{Figura 1}
        \end{figure}

		O jogo foi primeiramente implementado em uma versão jogador contra jogador, \emph{Player vs Player}, para verificar se a lógica implementada para o funcionamento estaria correta e concisa, dirimindo erros desnecessários na hora de implementar a árvore. Ao implementar o tabuleiro do jogo, foi decidido utilizar uma matriz 2 por 7, o que na visão dos desenvolvedores, facilita o entendimento, devido a sua disposição espacial ser equivalente ao do tabuleiro.

		Para realizar as manipulações do jogador foram criadas funções que controlam a lógica para ambos os jogadores, estas são, \emph{turno\_p1}, \emph{referencia}, \emph{turno\_p2}, \emph{referencia2}. Tais funções possuem uma lógica espelhada, sendo sua lógica geral praticamente imutável, isso decorre do fato do tabuleiro possuir dois lados de lógicas opostas, porém similares. \emph{referencia}, são funções criadas para adaptar a jogada, passada pelo usuário como uma letra, à lógica matricial, um número. Foi decidido o uso de letras para escolher a cavidade a ser jogada, para prevenir confusões entre a escolha da cavidade e a quantidade de sementes na mesma.

		Para o processo de finalização do jogo foram criadas outras duas funções, \emph{m\_vazia}, \emph{final}. Estas respectivamente, checa se algum dos lados dos jogadores esta vazio, e distribui as sementes restantes caso o jogo finalizado e mostra a mensagem de vitória ou derrota.

		Após todo esse processo para o funcionamento trivial do jogo ser criado, começou-se o desenvolvimento da Inteligência Artificial, que para facilitar o desenvolvimento da idéia será explicado em três partes, o processo de criação do nó, o processo de gerar a árvore e o retorno da jogada a ser feita.

		A estrutura básica do nó, foi recebendo parametros novos de acordo com a necessidade, no final do projeto obtivemos um nó com \emph{mat\_estado}, \emph{no *filhos}, \emph{heuristica}, \emph{player}, \emph{jogada}. \emph{mat\_estado} é a matriz do estado atual do jogo, \emph{no *filhos}, é um vetor de ponteiros de nós, após bastante debate, foi definido que esse seria o jeito mais prático e eficiente de definir os filhos, pois trabalhariamos apenas com as quantidades necessárias. \emph{heuristica}, é o valor da jogada, \emph{player}, quem está jogando no momento, jogador 1 ou jogador 2, e \emph{jogada} define qual a jogada foi feita para gerar o mat\_estado. Para realizar o preenchimento dos nós, foi feita a função \emph{criaNo}.

		Para implementar a função geradora da árvore implementamos uma função recursiva \emph{geraArvore} que recebe um nó previamente criado para servir de raiz de origem para a geração de outros nós. Incialmente definimos um critério de parada para o \emph{loop} recursivo, para isso utilizamos a variável \emph{dificuldade}. Ao entrar na função checa-se se a dificuldade é maior que zero, se sim, então prossegue-se com a criação de nós e para cada chamada da função \emph{dificuldade-1} é passado. Após passar por esse teste o algoritmo procede para um loop do tipo \emph{for} que testará através das possibilidades de jogadas, encontradas na matriz, se é possível criar um nó para aquele \emph{filho}, ou seja, se a casa no tabuleiro for diferente de zero, então cria-se um filho respectivo a posição daquela casa. Assim passamos a matriz do tabuleiro atual para uma \emph{mat\_aux} para que possamos simular para cada nó um jogada usando as funções \emph{turno\_p1} ou \emph{turno\_p2}. Para saber qual das funções anteriores utilizar para simular a jogada criamos um marcador global com o nome de \emph{flag} que assim sendo 1 ou 2 saberíamos qual delas utilizar. O último detalhe é a atribuição da heurística para cada nó. Toda vez que se cria um nó analisa-se a jogada anterior e a atual e subtrai-se resultando assim, na heurística.

		A função \emph{minimax} foi a última a ser implementada e passou por diversas mudanças. Incialmente é importante esclarecer que ela retorna um ponteiro do tipo nó. Ao entrar na função cada nó passa por um \emph{loop} do tipo \emph{for} onde se checa se o nó é uma folha ou não. Isso faz parte do primeiro teste da função, onde se testa se a \emph{dificuldade}, ou seja,  altura da árvore, é maior que zero e se o nó é uma folha. Caso passe no teste o procedimento levará para um outro \emph{loop} que irá passar por todos os nós e seus respectivos filhos aplicando a lógica da função \emph{Min} e \emph{Max}. Para isso foram criados duas funções auxiliares chamadas de \emph{Max} que retorna o maior dos valores e o \emph{Min} que retorna o menor dos valores. Ao final do processo o valor de retorno é um nó que possui a melhor heurística.

		Encerrando a lógica do modo \emph{Player vs IA} após a a escolha do nó com melhor heurística este representa um filho dentre os 6 criados para cada elemento, e como cada filho representa uma possibilidade de jogada automaticamente detectamos qual será a seleção de jogada e a introduzimos na função \emph{turno\_p2}. Por devir o \emph{loop} de chamar a \emph{geraArvore}, \emph{minmax} e selecionar a jogada se repete até o encerramento do jogo.

	\section{Conclusão}

		Árvore de jogos é uma alternativa viável para implementação de uma inteligência artificial básica, utilizando a possibilidade de calcular jogadas futuras para tentar prever as jogadas do inimigo e planejar suas próprias.

		Durante o desenvolvimento do trabalho foram encontrados vários problemas de implementação. Um dos motivos principais para tais problemas foi a possibilidade do jogador repetir sua jogada, gerando inconsistências na árvores e funções de avaliação.

	\begin{thebibliography}{9}

		\bibitem{gameTree}
	    Árvores de Jogos, https://en.wikipedia.org/wiki/Game\_tree

		\bibitem{siteAlch}
		Prof. Eduardo Alchieri, Estrutura de Dados, Slides, Árvores,\\ http://cic.unb.br/$\sim$alchieri/disciplinas/graduacao/ed/arvores.pdf

		\bibitem{mancalaOnline}
		Jogo Mancala Online, http://play-mancala.com

		\bibitem{mancala}
		Mancala, https://en.wikipedia.org/wiki/Mancala

		\bibitem{gobet}
		Gobet, F. (2009). ``Using a cognitive architecture for addressing the question of cognitive universals in cross-cultural psychology: The example of awalé''. Journal of Cross-Cultural Psychology 40 (4): 627–648. doi:10.1177/0022022109335186

	\end{thebibliography}


\end{document}
