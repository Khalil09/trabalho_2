\documentclass[a4paper, 12pt]{article}
\usepackage[portuguese]{babel}
\usepackage[utf8]{inputenc}
\usepackage[T1]{fontenc}
\usepackage{amssymb}
\usepackage{indentfirst}
\usepackage{graphicx}
\usepackage{fancyhdr}
\usepackage{circuitikz}
\usepackage{tikz}
\usepackage{authblk}
\usepackage{geometry}
\usepackage[colorinlistoftodos]{todonotes}
\usepackage{caption}
\usepackage{amsmath}
\usepackage{algorithm}
\usepackage[noend]{algpseudocode}
\geometry{a4paper, total={170mm,257mm}, left=25mm, top=25mm, right=25mm, bottom=25mm}
\usetikzlibrary{positioning}
\DeclareGraphicsExtensions{.pdf,.png,.jpg}
\makeindex
\author[]{Renato Nobre - 15/0146698\\Khalil Carsten - 15/0134495}
\affil[]{Estrutura de Dados\\Universidade de Brasília}
\title{Relatório\\Trabalho 2}
\date{21 de Junho de 2016}
\pagestyle{fancy}
\fancyhf{}
\lhead{Trabalho 2}
\rhead{Universidade de Brasília\\IE - Departamento de Ciência da Computação}
\headsep = 40pt
\captionsetup{labelformat=empty}

\makeatletter
\def\BState{\State\hskip-\ALG@thistlm}
\makeatother

\begin{document}
	\begin{titlepage}

		\newcommand{\HRule}{\rule{\linewidth}{0.5mm}}
		\centering
        %\maketitle
	    \textsc{\LARGE Universidade de Brasília}\\[0.5cm]
	    \includegraphics{logo.jpg}\\[0.5cm]
		\textsc{\Large Instituto de Ciências Exatas}\\[0.5cm]
	    \textsc{\Large Departamento de Ciência da Computação}\\[0.5cm]
		\textsc{\Large Estrutura de Dados - Turma "A"}\\[0.5cm]
	    \HRule \\[0.4cm]
	    { \huge \bfseries Relatório - Trabalho 2}\\
	      \HRule \\[1.5cm]
	      	\begin{minipage}{0.4\textwidth}
	      		\begin{flushleft} \large
	      			\emph{Nome:}\\
	     			\emph{Renato Nobre}\\
	      			\emph{Khalil Carsten}\\
	      		\end{flushleft}
	      	\end{minipage}
	      	~
	      	\begin{minipage}{0.4\textwidth}
	      		\begin{flushright} \large
	      			\emph{Matrícula:}\\
	     			\textsc{15/0146698}\\
	        		\textsc{15/0132662}\\
	      		\end{flushright}
	      	\end{minipage}\\[2cm]
		\textsc{\large \centering 03 de Março de 2016}\\
	\end{titlepage}
	
	
	\section{Introdução}
		\textrm{Mancala é denominada à uma categoria de jogos com sua evidencia de criação no século 6 e 7 \emph{Anno Domini}, havendo evidencias de sua criação em Eritreia e Etiopia. O Mancala apresenta claramente, similaridades com a agricultura, e a ausencia de equipamentos que podem ser evidencia do inicio da civilização em si [4]. Uma versão moderna do jogo é o Kalah, usualmente jogado nos Estados Unidos e na Europa, onde há a confusão entre as nomeclaturas do jogo. Como outros jogos de tabuleiro, o mancala já serviu de diversos estudos, tanto psicológicos, como na ciência da computação [5].}
		
		\textrm{O Kalah consiste de um tabuleiro de 14 cavidades, sendo 2 maiores denominadas Kahalas, e outras 12 cavidades menores, sendo 6 para cada jogador. A Kahala de cada jogador fica ao lado direito das cavidades menores, e o número de sementes define a pontuação do jogador.}
		
		\textrm{O estado inicial do jogo consiste em 4 sementes em cada cavidade dos jogadores, uma jogada é realizada quando o jogador escolhe uma das seis cavidades, retira todas as suas sementes e as distribui uma para cada cavidade no sentido anti-horário. A Kahala do jogador deve ser semeada também, entanto a Kahala do adversário deve ser ignorada. O jogo termina quando não há mais sementes em algum dos lados do tabuleiro. O vencedor é aquele que no final do jogo tiver mais sementes em sua Kahala. Regras adicionais são descritas detalhadamente dentro do programa.}
	
		\textrm{Em estrutura de dados, uma árvore é uma forma de organização hierárquica. Contendo, nós, raiz, ramos, folhas, e mais diversas terminologias para sua classificação. Tal estrutura é amplamente utilizada em classificações e tomada de decisões, como por exemplo no desenvolvimento de uma inteligência artificial básica para um jogo simples de tabuleiro.}
		
		\textrm{Uma árvore usualmente utilizada para esse propósito é a árvore genética, \emph{Game Tree}. Tal estrutura de ávore é montada de maneira a representar as possibilidades de jogadas de um jogador, a partir do estado do jogo. Os filhos de um nó representam todos os estados de jogada a partir da situação atual do jogo. Tal árvore pode ser criada com uma heuristica para cada nó, e com uma função de avaliação \emph{Minimax}, retornar a melhor jogada.}
		
		
	\section{Implementação}
		\textrm{Com o propósito de entender por total o funcionamento e regras do jogo, um tempo foi gasto realizando seu estudo. Para isso foi utilizado uma versão online do jogo, a mesma versão nos forneceu as regras, a quais foram utilizadas na seção \emph{Regras}, do programa. [3]}
		
		\textrm{O código é feito de maneira a tentar maximizar a separação da lógica da interface, contendo diversas funções para mostrar as mensagens na tela, disponibilizadas no começo do código, \emph{tabuleiro}, \emph{menu}, \emph{menu_dific}. Tais funções servem respectivamente para mostrar o tabuleiro na tela, mostrar o menu principal e o menu de regras, e mostrar o menu após o modo \emph{Player vs IA} ser escolhido. Há também uma função \emph{popular}, que preenche o tabuleiro com os valores iniciais do jogo.}
		
		\textrm{O jogo foi primeiramente implementado em uma versão jogador contra jogador, \emph{Player vs Player}, para verificar se a lógica implementada para o funcionamento estaria correta e concisa, dirimindo erros desnecessários na hora de implementar a árvore. Ao implementar o tabuleiro do jogo, foi decidido utilizar uma matriz 2 por 7, o que na visão dos desenvolvedores, facilíta o entendimento, devido a sua disposição espacial ser equivalente ao do tabuleiro.}
		
		\textrm{Para realizar as manipulações do jogador foram criadas funções que controlam a lógica para ambos os jogadores, estas são, \emph{turno_p1}, \emph{referencia}, \emph{turno_p2}, \emph{referencia2}. Tais funções possuem uma lógica espelhada, sendo sua lógica geral práticamente imutavel, isso decorre do fato do tabuleiro possuir dois lados de lógicas opostas, porém similares. \emph{referencia}, são funções criadas para adaptar a jogada, passada pelo usuário como uma letra, à lógica matricial, um número. Foi decidido o uso de letras para escolher a cavidade a ser jogada, para prevenir confusões entre a escolha da cavidade e a quantidade de sementes na mesma.}
		
		\begin{algorithm}
		\caption{turno_px}\label{euclid}
		\begin{algorithmic}[1]
		\Procedure{$turno_px(matriz, escolha, jogada_AI)$}
			\State{$tabuleiro(matriz)$}
			\If{(simulando = FALSE)} 
				\State{Seleciona Jogada}
			\Else
				\State $jogada \gets jogada_AI$
			\Endif
			\State $n \gets matriz[x][jogada]$
			\State $matriz[x][jogada] \gets 0$
			\While{$n > 0$}
				\State{Destribui as Sementes}
			\EndWhile
			\If{captura sementes = TRUE}
				\State{Incrementa Total}
				\State{Zera a Cavidade do Oponente}
				\State{Zera a Cavidade do Jogador}
			\EndIf
			\If{Ultima Semente Cai na Kahala}
				\State{$tabuleiro(matriz)$}
				\State{$turno_px(matriz, escolha, jogada_AI)$}
			\EndIf
		\EndProcedure
		\end{algorithmic}
		\end{algorithm}
		
		\textrm{Para o processo de finalização do jogo foram criadas outras duas funções, \emph{m_vazia}, \emph{final}. Estas respectivamente, checa se algum dos lados dos jogadores esta vazio, e distribui as sementes restantes caso o jogo finalizado e mostra a mensagem de vitória ou derrota.}

		\textrm{Após todo esse processo para o funcionamento trivial do jogo ser criado, começou-se o desenvolvimento da Inteligência Artíficial, que para facilitar o desenvolvimento da idéia será explicado em três partes, o processo de criação do nó, o processo de gerar a árvore e o retorno da jogada a ser feita.}
		
		\textrm{A estrutura básica do nó, foi recebendo parametros novos de acordo com a necessidade, no final do projeto obtivemos um nó com \emph{mat_estado}, \emph{no *filhos}, \emph{heuristica}, \emph{player}, \emph{jogada}. \emph{mat_estado} é a matriz do estado atual do jogo, \emph{no *filhos}, é um vetor de ponteiros de nós, após bastante debate, foi definido que esse seria o jeito mais prático e eficiente de definir os filhos, pois trabalhariamos apenas com as quantidades necessárias. \emph{heuristica}, é o valor da jogada, \emph{player}, quem está jogando no momento, jogador 1 ou jogador 2, e \emph{jogada} define qual a jogada foi feita para gerar o mat_estado. Para realizar o preenchimento dos nós, foi feita a função \emph{criaNo}.}
	
		\textrm{} %Falar sobre a gera arvore
		
		\texrm{} %falar sobre a minimax e suas dependencias

	\section{Conclusão}
		\textrm{Árvore de jogos é uma alternativa viavel para implementação de uma inteligencia artificial básica, utilizando a possibilidade de cálcular jogadas futuras para tentar prever as jogadas do inimigo e planejar suas próprias.}
		
		\textrm{Durante o desenvolvimento do trabalho foram encontrados vários problemas de implementação. Um dos motivos principais para tais problemas foi a possibilidade do jogador repetir sua jogada, gerando inconsistencias na árvores e funções de avaliação.}
		
	\section{Bibliografia}
	    [1] Árvores de Jogos, https://en.wikipedia.org/wiki/Game_tree
	
		[2] Prof. Eduardo Alchieri, Estrutura de Dados, Slides, Árvores, http://cic.unb.br/~alchieri/disciplinas/graduacao/ed/arvores.pdf 
		
		[3] Jogo Mancala Online, http://play-mancala.com
		
		[4] Mancala, https://en.wikipedia.org/wiki/Mancala
	
		[5] Gobet, F. (2009). '"Using a cognitive architecture for addressing the question of cognitive universals in cross-cultural psychology: The example of awalé". Journal of Cross-Cultural Psychology 40 (4): 627–648. doi:10.1177/0022022109335186
		
		
		
\end{document}